\documentclass[load-preamble+,scrartcl={DIV10}]{cnltx-doc}
\usepackage[utf8]{inputenc}
\usepackage{carbohydrates}

\setcnltx{
  package  = {carbohydrates} ,
  authors  = Clemens Niederberger ,
  email    = contact@mychemistry.eu ,
  info     = {carbohydrate molecules with \pkg*{chemfig}} ,
  url      = https://github.com/cgnieder/carbohydrates/ ,
  abstract = {%
    \carbohydrates\ offers macros for making exercise sheets when teaching
    carbohydrate chemistry a lot less tedious.  It uses \pkg{chemfig} for
    drawing the formulae.
  } ,
  add-cmds = {
    allose , altrose , arabinose ,
    carbohydrate ,
    desoxyribose ,
    erythrose ,
    galactose , glucose , glycerinaldehyde , gulose ,
    idose ,
    lyxose ,
    mannose ,
    newaldose ,
    ribose ,
    setcarbohydrate , setcarbohydratedefaults ,
    talose , threose ,
    xylose
  } ,
  add-silent-cmds = {
    bondboldwidth , bondhashlength , bondwidth ,
    D ,
    chemfig , chemname , chemnameinit ,
    definesubmol ,
    iupac ,
    printatom ,
    setatomsep , setbondoffset , setbondstyle , setcrambond , setdoublesep ,
  }
}

\defbibheading{bibliography}[References]{\addsec{#1}}
\addbibresource{\jobname.bib}

\usepackage{filecontents}
\begin{filecontents*}{\jobname.bib}
@online{texdev,
  url = {http://www.texdev.net/} ,
  urldate = {2014-04-25} ,
  author = {Joseph Wright}
}
\end{filecontents*}

\usepackage{chemmacros}

\newcommand*{\bondwidth}{0.06642 em}
\newcommand*{\bondboldwidth}{0.22832 em}
\newcommand*{\bondhashlength}{0.25737 em}
\setdoublesep{0.35700 em}
\setatomsep{1.78500 em}
\setbondoffset{0.18265 em}
\setbondstyle{line width = \bondwidth}
\setcrambond
  {\the\dimexpr \bondwidth * 2 + \bondboldwidth \relax}
  {\bondwidth}{\bondhashlength}
\renewcommand*\printatom[1]{{\small\ensuremath{\mathsf{#1}}}}

\usepackage{rotating,array,tabu,booktabs}

\defabbr\vs{\cnltxlatin{vs}}

\begin{document}

\section{Licence and Requirements}
\license

\carbohydrates\ loads the packages \pkg{chemfig}~\cite{pkg:chemfig} and
\pkg*{cnltx-base}, the programming package fom the \bnd{cnltx}
bundle~\cite{bnd:cnltx}.

\section{The Idea}
When teaching chemistry -- and in the case of this package: carbohydrate
chemistry -- you have to show many examples of the basic aldoses and you have
to explain the Fischer and the Haworth representation as well. This means you
may have nearly the same \pkg{chemfig} formulae over and over in your
documents.  \carbohydrates\ aims to ease this task.

The following example will give a short impression of what the package does:
\begin{example}
  \glucose
  \glucose[model={fischer=skeleton}]
  \setatomsep{2.5em}
  \glucose[model=haworth,ring]
  \glucose[model=haworth,ring=furanose]
\end{example}

\section{Usage}
\subsection{The Base Macro}
\begin{commands}
  \command{carbohydrate}[\oarg{options}\marg{spec}]
    A generic macro for typesetting carbohydrates.
\end{commands}
We will talk about the options in a bit. First lets see what \meta{spec}
means.  This argument is a series of the tokens \code{r}, \code{l} and
\code{0} denoting an hydroxy group placed to the right or the left in the
Fischer projection of the carbohydrate.  A \code{0} means thet the hydroxy
group is to be left out.  The series of tokens is meant to describe the
hydroxy groups at the chiral centers. The aldehyde group\footnote{Also the
  keto group once ketoses will be implemented.} and the hydrohy group at the
end of the molecule will be set automatically.  Unless specified otherwise the
command expects a hexose which means a series of \emph{four} tokens.

\begin{example}
  \chemname{\carbohydrate{llrr}}{\iupac{\D-Mannose}}
  \chemname{\carbohydrate[model={fischer=skeleton}]{llrr}}{\iupac{\D-Mannose}}
  \setatomsep{2.5em}\chemnameinit{}
  \chemname{\carbohydrate[model=chair]{llrr}}{\iupac{\D-Mannose}}
  \chemname{\carbohydrate[model=haworth]{llrr}}{\iupac{\D-Mannose}}
\end{example}

Adding the option \option{pentose} means that now only \emph{three} tokens
need to be specified.

\begin{example}
  \chemname{\carbohydrate[pentose]{rlr}}{\iupac{\D-Xylose}}
  \chemname{\carbohydrate[pentose,model={fischer=skeleton}]{rlr}}{\iupac{\D-Xylose}}
  \setatomsep{2.5em}\chemnameinit{}
  \chemname{\carbohydrate[pentose,model=haworth,ring]{rlr}}{\iupac{\D-Xylose}}
  \chemname{\carbohydrate[pentose,model=haworth,ring=pyranose]{rlr}}{\iupac{\D-Xylose}}
\end{example}

\subsection{Available Options}
As you have already seen in the pevious examples \cs{carbohydrate} has an
optional argument that takes different options.  Here is a complete list:
\begin{options}
  % model/fischer/skeleton/.code = \def\cbhdr@model{fischer@skeleton} ,
  % model/fischer/full/.code     = \def\cbhdr@model{fischer} ,
  \keychoice{model}{fischer,haworth,chair}\Default{fischer}
    The model to be used to draw the molecule.  The choice \option{fischer}
    is itself an option with two choices: \keyis{fischer}{skeleton} and
    \keyis{fischer}{full}.  Leaving the choice out will use \code{full} as
    default choice.
  \opt{chain}
    Draw the open chain isomer.
  \keychoice{ring}{\default{true},pyranose,furanose}
    Draw a ring isomer.  If you don't specify what ring type should be draw
    (\ie, if you choose \code{true}) the default depends on the length of the
    carbohydrate.  Ror example for hexoses the default ring type is
    \code{pyranose}.
  \keychoice{anomer}{alpha,beta,undetermined}\Default{alpha}
    The ring anomer.
  \keychoice{length}{6,5,4,3}\Default{6}
    The length of the carbohydrate.  \keyis{length}{6} draws a hexose,
    \keyis{length}{3} draws a triose.
  \opt{hexose}
    An alias for \keyis{length}{6}.
  \opt{pentose}
    An alias for \keyis{length}{5}.
  \opt{tetrose}
   An alias for \keyis{length}{4}.
  \opt{triose}
    An alias for \keyis{length}{3}.
  \keybool{3d}\Default{false}
    Draw some of the bonds of the rings in the \code{haworth} and \code{chair}
    in a way that indicates the three dimensional structure of the molecules.
\end{options}

\subsection{Defining Shortcuts}\label{sec:defining-shortcuts}

\carbohydrates\ allows to define shortcuts for aldoses:

\begin{commands}
  \command{newaldose}[\marg{cs}\oarg{options}\marg{spec}]
    This defines the macro \meta{cs} with preset options \meta{options}.
    \meta{spec} has the same meaning as for \cs{carbohydrate}.
  \command{renewaldose}[\marg{cs}\oarg{options}\marg{spec}]
    The same command but redefines an existing macro.
\end{commands}

In fact, \carbohydrates\ already defines macros for the common aldoses.  They
are listed in table~\ref{tab:predefined-aldoses} on
page~\pageref{tab:predefined-aldoses}.  They don't have any predefined options
(except for \option{hexose}, \option{pentose} \etc).

\begin{sidewaystable}
  \setcarbohydrate{model={fischer=skeleton}}
  \caption{Overview over the predefined aldoses.}
  \label{tab:predefined-aldoses}
  \begin{tabu}{*{8}{X[c]<{\strut}}}
    \toprule
      \allose & \altrose & \glucose   & \mannose &
      \gulose & \idose   & \galactose & \talose \\
      \cs{allose} & \cs{altrose} & \cs{glucose}   & \cs{mannose} &
      \cs{gulose} & \cs{idose}   & \cs{galactose} & \cs{talose} \\
      \iupac{\D\-Allose}    & \iupac{\D\-Altrose} &
      \iupac{\D\-Glucose}   & \iupac{\D\-Mannose} &
      \iupac{\D\-Gulose}    & \iupac{\D\-Idose} &
      \iupac{\D\-Galactose} & \iupac{\D\-Talose} \\
    \midrule
      \ribose & \arabinose & \xylose & \lyxose & \desoxyribose \\
      \cs{ribose} & \cs{arabinose} & \cs{xylose} & \cs{lyxose} &
      \multicolumn{2}{l}{\cs{desoxyribose}} \\
      \iupac{\D\-Ribose} & \iupac{\D\-Arabinose} &
      \iupac{\D\-Xylose} & \iupac{\D\-Lyxose} &
      \iupac{\D\-Desoxy\|ribose} \\
    \midrule
      \erythrose & \threose & \glycerinaldehyde \\
      \cs{erythrose} & \cs{threose} & \multicolumn{2}{l}{\cs{glycerinaldehyde}} \\
      \iupac{\D\-Erythrose} & \iupac{\D\-Threose} &
      \iupac{\D\-Glycerin\|aldehyde} \\
    \bottomrule
  \end{tabu}
\end{sidewaystable}

\subsection{Available Models}\label{sec:available-models}

\carbohydrates\ implements different models how to draw carbohydrates:
\begin{itemize}
  \item Fischer -- skeleton: the Fischer representation with only a skeleton
    formula.
  \item Fischer -- full: the Fischer representation including all C and H
    atoms.
  \item Haworth: the Haworth representation.
  \item Chair: the chair conformation.
\end{itemize}

While the Fischer model is implemented for all carbohydrates both Haworth and
chair are not.  The chair model is only implemented for aldohexoses, Haworth
is implemented for aldotetroses, -pentoses and -hexoses.

\begin{example}
  \glucose[model={fischer=skeleton}]
  \glucose[model={fischer=full}]
  \setatomsep{2.5em}
  \glucose[model=haworth]
  \glucose[model=chair]
\end{example}

\subsection{Chain \vs\ Ring Forms}

While the chain forms are available in all models the ring forms obviously
aren't.  There are two ring forms for hexoses and pentoses: pyranoses and
furanoses.  For tetroses only the furanose rings are available as there don't
exist pyranose ring forms (for obvious reasons).  It is also clear that
neither pyranose nor furanose forms of trioses exist.

\begin{example}
  \setatomsep{2.5em}
  \glucose[model=haworth,ring]
  \ribose[model=haworth,ring]
  \threose[model=haworth,ring]
\end{example}

Actually the above is not true: the chain forms are not available in all
models for all aldoses.  As said in section~\ref{sec:available-models} the
chair model is only implemented for aldohexoses.  Also for chains are not
implemented for tetroses and trioses in the Haworth model.

\subsection{Default Settings}
\begin{commands}
  \command{setcarbohydrate}[\marg{options}]
    Set package options for all carbohydrates within the current scope.
  \command{setcarbohydratedefaults}[\marg{csname}\marg{options}]
    Set options for a predefined carbohydrate within the current scope. The
    first argument \meta{csname} is the macro name of the shortcut (see
    section~\ref{sec:defining-shortcuts}) \emph{without} leading backslash.
\end{commands}

\begin{example}
  \setatomsep{2.5em}
  \setcarbohydratedefaults{glucose}{ring,model=haworth,anomer=undetermined}
  \glucose\
  \mannose
\end{example}


\section{Restrictions and TODOs}
\subsection{TODOs}

There are still a number of missing features that I plan to implement
someday such as:
\begin{itemize}
  \item ring forms for \laevus-carbohydrates,
  \item support for ketoses and
  \item support for oxidized and reduzed forms.
\end{itemize}

\section{About the Examples}
All macros used in the examples either belong to \carbohydrates\ and are
described in this manual or they belong to either \pkg{chemfig} or
\bnd{chemmacros}~\cite{bnd:chemmacros}\footnote{That is if they're not
  standard \LaTeX\ macros.}.  I encourage you to take a look at those packages
if you don't know them already.

The \pkg{chemfig} settings have been adjusted for the examples in this
manual.  Specifically the preamble of this document makes these settings:

\begin{sourcecode}
  \newcommand*{\bondwidth}{0.06642 em}
  \newcommand*{\bondboldwidth}{0.22832 em}
  \newcommand*{\bondhashlength}{0.25737 em}
  \setdoublesep{0.35700 em}
  \setatomsep{1.78500 em}
  \setbondoffset{0.18265 em}
  \setbondstyle{line width = \bondwidth}
  \setcrambond
    {\the\dimexpr \bondwidth * 2 + \bondboldwidth \relax}
    {\bondwidth}{\bondhashlength}
  \renewcommand*\printatom[1]{{\small\ensuremath{\mathsf{#1}}}}
\end{sourcecode}
These settings are taken from~\cite{texdev}.  Search the page for
\code{chemfig} and you should be able to find them there.

\end{document}
