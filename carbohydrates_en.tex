\documentclass[load-preamble+,scrartcl={DIV10}]{cnltx-doc}
\usepackage[utf8]{inputenc}
\usepackage{carbohydrates}

\setcnltx{
  package  = {carbohydrates} ,
  authors  = Clemens Niederberger ,
  email    = contact@mychemistry.eu ,
  info     = {carbohydrate molecules with \pkg*{chemfig}} ,
  url      = https://github.com/cgnieder/carbohydrates/ ,
  abstract = {%
    \carbohydrates\ offers macros for making exercise sheets when teaching
    carbohydrate chemistry a lot less tedious.  It uses \pkg{chemfig} for
    drawing the formulae.
  } ,
  add-cmds = {
    allose ,
    altrose ,
    arabinose ,
    carbohydrate ,
    desoxyribose ,
    erythrose ,
    galactose ,
    glucose ,
    glycerinaldehyde ,
    gulose ,
    idose ,
    lyxose ,
    mannose ,
    newaldose ,
    ribose ,
    setcarbohydrate ,
    setcarbohydratedefaults ,
    talose ,
    threose ,
    xylose
  } ,
  add-silent-cmds = {
    D ,
    chemfig ,
    chemname ,
    chemnameinit ,
    definesubmol ,
    iupac ,
    setatomsep
  }
}

\defbibheading{bibliography}[References]{\addsec{#1}}

\usepackage{chemmacros}

\newcommand{\bondwidth}{0.06642 em} % 'Line Width'
\newcommand*{\bondboldwidth}{0.22832 em} %'Bold Width'
\newcommand*{\bondhashlength}{0.25737 em} % 'Hash Spacing'
\setdoublesep{0.35700 em}  % 'Bond Spacing'
\setatomsep{1.78500 em}    % 'Fixed Length'
\setbondoffset{0.18265 em} % 'Margin Width'
\setbondstyle{line width = \bondwidth}
\setcrambond
  {\the\dimexpr \bondwidth * 2 + \bondboldwidth \relax}
  {\bondwidth}{\bondhashlength}
\renewcommand*\printatom[1]{{\small\ensuremath{\mathsf{#1}}}}

\begin{document}

\section{Licence and Requirements}
\license

\section{The Idea}
When teaching chemistry -- and in the case of this package: carbohyddrate
chemistry -- you have to show many examples of the basic aldoses and you have
to explain the Fischer and the Haworth representation as well. This means you
may have nearly the same \pkg{chemfig}~\cite{pkg:chemfig} formulae over and
over in your documents.  \carbohydrates\ aims to ease this task.

The following example will give a short impression of what the package does:
\begin{example}
  \glucose
  \glucose[model={fischer=skeleton}]
  \setatomsep{2.5em}
  \glucose[model=haworth,ring]
  \glucose[model=haworth,ring=furanose]
\end{example}

\section{The Basic Macro}
\begin{commands}
  \command{carbohydrate}[\oarg{options}\marg{spec}]
    A generic macro for typesetting carbohydrates.
\end{commands}
We will talk about the options in a bit. First lets see what \meta{spec}
means.  This argument is a series of the tokens \code{r}, \code{l} and
\code{0} denoting an hydroxy group placed to the right or the left in the
Fischer projection of the carbohydrate.  A \code{0} means thet the hydroxy
group is to be left out.  The series of tokens is meant to describe the
hydroxy groups at the chiral centers. The aldehyde group\footnote{Also the
  keto group once ketoses will be implemented.} and the hydrohy group at the
end of the molecule will be set automatically.  Unless specified otherwise the
command expects a hexose which means a series of \emph{four} tokens.

\begin{example}
  \chemname{\carbohydrate{llrr}}{\iupac{\D-Mannose}}
  \chemname{\carbohydrate[model={fischer=skeleton}]{llrr}}{\iupac{\D-Mannose}}
  \setatomsep{2.5em}\chemnameinit{}
  \chemname{\carbohydrate[model=chair]{llrr}}{\iupac{\D-Mannose}}
  \chemname{\carbohydrate[model=haworth]{llrr}}{\iupac{\D-Mannose}}
\end{example}

Adding the option \option{pentose} means that now only \emph{three} tokens
need to be specified.

\begin{example}
  \chemname{\carbohydrate[pentose]{rlr}}{\iupac{\D-Xylose}}
  \chemname{\carbohydrate[pentose,model={fischer=skeleton}]{rlr}}{\iupac{\D-Xylose}}
  \setatomsep{2.5em}\chemnameinit{}
  \chemname{\carbohydrate[pentose,model=haworth,ring]{rlr}}{\iupac{\D-Xylose}}
  \chemname{\carbohydrate[pentose,model=haworth,ring=pyranose]{rlr}}{\iupac{\D-Xylose}}
\end{example}

\section{Restrictions}

\end{document}
